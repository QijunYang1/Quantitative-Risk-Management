\documentclass[11pt,en]{elegantpaper}
\title{Quantitative Risk Management Project 7}
\author{Qijun Yang \\ Duke University}
\institute{\href{https://fintech.meng.duke.edu}{Financial Technology at Duke University}}
\version{1.0}
\date{Mar. 25, 2023}

% cmd for this doc
\usepackage{array}
\usepackage{listings}
\usepackage{graphicx}
\usepackage{subfigure}
\usepackage{multirow}
\newcommand{\ccr}[1]{\makecell{{\color{#1}\rule{1cm}{1cm}}}}

\addbibresource[location=local]{reference.bib} % reference file
\begin{document}
\maketitle

\section*{\textcolor{orange}{Problem 1}}

\begin{itemize}
    \item Current Stock Price \$165
    \item Strike Price \$165
    \item Current Date 03/13/2022
    \item Options Expiration Date 04/15/2022
    \item Risk Free Rate of 4.25\%
    \item Continuously Compounding Coupon of 0.53\%
    \item implied volatility 20\%
\end{itemize}

Implement the closed form greeks for GBSM. Implement a finite difference derivative calculation.

Compare the values between the two methods for both a call and a put.

Implement the binomial tree valuation for American options with and without discrete dividends. 

Assume the stock above:

\begin{itemize}
    \item Pays dividend on 4/11/2022 of \$0.88
\end{itemize}

Calculate the value of the call and the put. Calculate the Greeks of each.


\section*{\textcolor{orange}{Answer}}

As to first problem, I compare the effect of the greeks closed form of Gengeralized Black Scholes Model and  finite difference methods for European options. The results are following:

\begin{table}[htbp]
    \centering
    \caption{Greeks: Closed Form Vs. Finite Differnece}
    \begin{tabular}{@{}cccccccc@{}}
        \toprule
        \textbf{} &  & \textbf{Detla} & \textbf{Gamma} & \textbf{Vega} & \textbf{Theta} & \textbf{Rho} & \textbf{Carry Rho} \\
        \midrule
        Closed Form & Call & 0.534009 & 0.040038 & 19.71018 & -24.89852 & 7.583586 & 7.966246 \\
        & Put & -0.465512 & 0.040038 & 19.71018 & -18.787 & -7.277011 & -6.944416 \\
       Finite Difference & Call & 0.534009 & 0.040038 & 19.7101 & -24.93221 & 7.583554 & 7.966254 \\
        & Put & -0.465512 & 0.040038 & 19.7101 & -18.82068 & -7.277045 & -6.944409\\
        \bottomrule
    \end{tabular}
\end{table}

\newpage

As we could see, the greeks of European options from cloesd form and finite difference is very similar. The difference between the two is very small. 

Use the binomial tree methods to calculate the value and greeks of call and put American options(consider whether there is continuously compounding coupon or not):

\begin{table}[htbp]
    \centering
    \caption{Greeks: American Option with Continuously Compounding Coupon}
    \begin{tabular}{@{}ccccc@{}}
    \toprule
                                                                                       & \multicolumn{2}{c}{Without Discrete Dividend} & \multicolumn{2}{c}{With Discrete Dividend} \\
                                                                                       & Call                  & Put                   & Call                 & Put                 \\
                                                                                       \midrule
    \textbf{Valuation}                                                                 & 4.227506              & 3.714324              & 4.074976             & 4.147062            \\
    \textbf{Detla}                                                                     & 0.533966              & -0.473668             & 0.530507             & -0.495856           \\
    \textbf{Gamma}                                                                     & 0.280568              & 0.246837              & 0.031063             & 0.002387            \\
    \textbf{Vega}                                                                      & 19.685473             & 19.655211             & 19.576769            & 19.815527           \\
    \textbf{Theta}                                                                     & -24.904878            & -19.299822            & -24.464277           & -18.948359          \\
    \textbf{Rho}                                                                       & 7.583365              & -5.980132             & 6.797344             & -7.248928           \\
    \textbf{Carry Rho}                                                                 & 7.965619              & -5.738138             & 7.123061             & -6.913934           \\
    \textbf{\begin{tabular}[c]{@{}c@{}}Sensitivity to Dividend Amount\end{tabular}} & NaN                   & NaN                   & -0.115203            & 0.515291            \\ \bottomrule
    \end{tabular}
\end{table}

\begin{table}[htbp]
    \centering
    \caption{Greeks: American Option without Continuously Compounding Coupon}
    \begin{tabular}{@{}ccccc@{}}
    \toprule
                                                                                       & \multicolumn{2}{c}{Without Discrete Dividend} & \multicolumn{2}{c}{With Discrete Dividend} \\
                                                                                       & Call                  & Put                   & Call                 & Put                 \\
                                                                                       \midrule
                                                                                       & Call                                        & Put                                        & Call                                      & Put                                       \\
                                                                                       \textbf{Valuation}                                                                 & 4.269859                                    & 3.684138                                   & 4.112836                                  & 4.110535                                  \\
                                                                                       \textbf{Detla}                                                                     & 0.537384                                    & -0.471987                                  & 0.532904                                  & -0.493280                                 \\
                                                                                       \textbf{Gamma}                                                                     & 0.280501                                    & 0.242952                                   & 0.021277                                  & 0.003252                                  \\
                                                                                       \textbf{Vega}                                                                      & 19.680769                                   & 19.642167                                  & 19.574566                                 & 19.824035                                 \\
                                                                                       \textbf{Theta}                                                                     & -25.388799                                  & -18.988711                                 & -24.897100                                & -18.573133                                \\
                                                                                       \textbf{Rho}                                                                       & 7.630524                                    & -5.896065                                  & 6.834995                                  & -7.202673                                 \\
                                                                                       \textbf{Carry Rho}                                                                 & 8.016606                                    & -5.659202                                  & 7.163720                                  & -6.870814                                 \\
                                                                                       \textbf{\begin{tabular}[c]{@{}c@{}}Sensitivity to Dividend Amount\end{tabular}} & NaN                                         & NaN                                        & -0.115497                                 & 0.512470                                  \\
    \bottomrule
    \end{tabular}
\end{table}

We vary the changing amount of all variable to better make our sensitivity reasonable. Finally, we set the amount the price change is \$0.2 and all the other changing amount is 0.01. 




\newpage
\section*{\textcolor{orange}{Problem 2}}

Using the options portfolios from Problem3 last week (named problem2.csv in this week’s repo) and
assuming :

\begin{itemize}
    \item American Options
    \item Current Date 03/03/2023
    \item Current AAPL price is 151.03
    \item Risk Free Rate of 4.25\%
    \item Dividend Payment of \$1.00 on 3/15/2023
\end{itemize}

Using DailyPrices.csv. Fit a Normal distribution to AAPL returns 
\begin{itemize}
    \item  assume 0 mean return.
\end{itemize}

 Simulate AAPL returns 10 days ahead and apply those returns to the current AAPL price (above). Calculate Mean, VaR and ES.

Calculate VaR and ES using Delta-Normal.

Present all VaR and ES values a \$ loss, not percentages.

Compare these results to last week's results.

    
\section*{\textcolor{orange}{Answer}}

Use Normal distribution to simulate the returns 10 days ahead, apply those returns to the current AAPL price and we get the following:

\newpage
\begin{table}[htbp]
    \centering
    \caption{Normal Simulated Portfolio Value 10 Days Later For American Option}
    \begin{tabular}{@{}ccccc@{}}
    \toprule
                 & Mean(Portfolio Value) & Mean(Change) & VaR       & ES        \\ \midrule
    Straddle     & 13.573306             & 2.034698     & 1.175029  & 1.221071  \\
    SynLong      & 1.756892              & -0.163474    & 17.931696 & 22.04644  \\
    CallSpread   & 4.503748              & 0.034049     & 3.760683  & 4.071643  \\
    PutSpread    & 3.426409              & 0.508251     & 2.670079  & 2.794192  \\
    Stock        & 151.392764            & 0.362764     & 16.740359 & 20.588025 \\
    Call         & 7.665099              & 0.935612     & 5.97523   & 6.309565  \\
    Put          & 5.908207              & 1.099086     & 4.495994  & 4.656884  \\
    CoveredCall  & 146.359195            & -0.595879    & 12.87181  & 16.616972 \\
    ProtectedPut & 155.276754            & 1.22444      & 7.485625  & 7.832187  \\ \bottomrule
    \end{tabular}
    \end{table}


\begin{table}[htbp]
    \centering
    \caption{AR1 Simulated Portfolio Value 10 Days Later For European Option}
    \begin{tabular}{@{}ccccc@{}}
        \toprule
        \textbf{Portfolio} & \textbf{Mean(Portfolio Value)} & \textbf{Mean(Variation)} & \textbf{VaR} & \textbf{ES}\\
        \midrule
        Straddle     &             13.236627 &     1.586627 &   1.378317 &   1.387136 \\
        SynLong      &              2.050838 &     0.100838 &   16.24976 &  19.975054 \\
        CallSpread   &              4.519231 &    -0.070769 &    3.89146 &   4.181353 \\
        PutSpread    &              3.292349 &     0.282349 &     2.6562 &   2.811006 \\
        Stock        &            151.334622 &     0.304622 &  16.007244 &  19.710478 \\
        Call         &              7.643733 &     0.843733 &   6.039037 &   6.362422 \\
        Put          &              5.592895 &     0.742895 &   4.401389 &   4.601255 \\
        CoveredCall  &            146.323396 &    -0.656604 &  12.191148 &  15.780425 \\
        ProtectedPut &            154.976053 &     0.936053 &   8.074428 &   8.681969 \\
        \bottomrule
    \end{tabular}
\end{table}

Here, we could find that almost all the portfolio of American Options values more than European Options. The Sensitivity between the two is very small.


\section*{\textcolor{orange}{Problem 3}}

Use the Fama French 3 factor return time series (F-F\_Research\_Data\_Factors\_daily.CSV) as well as the Carhart Momentum time series (F-F\_Momentum\_Factor\_daily.CSV) to fit a 4 factor model to the following stocks.

\begin{table}[htbp]
    \centering
    \begin{tabular}{|l|l|l|l|}
    \hline
    AAPL  & FB    & UNH & MA   \\ \hline
    MSFT  & NVDA  & HD  & PFE  \\ \hline
    AMZN  & BRK-B & PG  & XOM  \\ \hline
    TSLA  & JPM   & V   & DIS  \\ \hline
    GOOGL & JNJ   & BAC & CSCO \\ \hline
    \end{tabular}
\end{table}

Fama stores values as percentages, you will need to divide by 100 (or multiply the stock returns by 100) to get like units.

Based on the past 10 years of factor returns, find the expected annual return of each stock.

Construct an annual covariance matrix for the 10 stocks.

Assume the risk free rate is 0.0425. Find the super efficient portfolio.

\section*{\textcolor{orange}{Answer}}

Use Fama French 3 factor and Carhart Carhart Momentum to fit the 4 factor model and based on the past 10 years of factor returns, we get the expected annual return of each stock like following:


\begin{table}[htbp]
    \centering
    \begin{tabular}{|c|c|c|c|}
    \hline
    AAPL    & META    & UNH     & MA      \\ \hline
    23.06\% & 7.65\%  & 20.39\% & 19.70\% \\ \hline
    MSFT    & NVDA    & HD      & PFE     \\ \hline
    22.39\% & 22.36\% & 16.92\% & 15.63\% \\ \hline
    AMZN    & BRK-B   & PG      & XOM     \\ \hline
    33.89\% & 14.39\% & 17.60\% & 8.61\%  \\ \hline
    TSLA    & JPM     & V       & DIS     \\ \hline
    11.82\% & 13.41\% & 18.70\% & 11.94\% \\ \hline
    GOOGL   & JNJ     & BAC     & CSCO    \\ \hline
    29.18\% & 11.21\% & 13.59\% & 21.78\% \\ \hline
    \end{tabular}
\end{table}

Then we optimize the portfolio sharpe ratio to find the super efficient portfolio.

The best sharpe ratio is 0.832215456951802

Here is the weight of each asset:

\begin{table}[htbp]
    \centering
    \begin{tabular}{|l|l|l|l|l|}
    \hline
           & \textbf{AAPL}  & \textbf{META}  & \textbf{UNH} & \textbf{MA}   \\ \hline
    Weight & 0              & 0              & 37\%         & 0             \\ \hline
           & \textbf{MSFT}  & \textbf{NVDA}  & \textbf{HD}  & \textbf{PFE}  \\ \hline
    Weight & 0              & 0              & 0            & 0             \\ \hline
           & \textbf{AMZN}  & \textbf{BRK-B} & \textbf{PG}  & \textbf{XOM}  \\ \hline
    Weight & 10\%           & 0              & 28\%         & 0             \\ \hline
           & \textbf{TSLA}  & \textbf{JPM}   & \textbf{V}   & \textbf{DIS}  \\ \hline
    Weight & 0              & 0              & 0            & 0             \\ \hline
           & \textbf{GOOGL} & \textbf{JNJ}   & \textbf{BAC} & \textbf{CSCO} \\ \hline
    Weight & 17\%           & 0              & 0            & 8\%          \\ \hline
    \end{tabular}
\end{table}
 

\end{document}